\documentclass[../thesis.tex]{subfiles}
\begin{document}
\chapter*{Abstract}

Algorithmic Trading (AT) is a financial sector that trades financial instruments, such as stocks, with algorithms and no human interaction. This allows the largest prop-trading firms in the world to conduct thousands of trades per second. In practice, the vast majority of strategies are implemented using mathematical formulas based on a variety of stock metrics, such as closing price or volume. However, the effectiveness of these algorithms aren't publicly available due to the necessity of secrecy of implementation details. Part of the thesis aims to uncover and examine the effectiveness of existing metrics-based strategies. We find both Pairs Trading and a combined RSI and MACD momentum algorithm to be incredibly effective.

Moving beyond traditional AT strategies, this thesis further aims to investigate using news- or media-content-based strategies. By using Twitter data and Natural Language Processing (NLP), we create a unique trading strategy based on Twitter sentiment of publicly traded companies tweets using a bevy of machine learning algorithms and a deep learning algorithm. We use models which have basic, extended, and stacked features. We find the basic model to be very ineffective while the extended model beats the baseline measure for 87.5\% of stocks tested generating profits of up to 629\%. Even though the stacked model only beats the baseline for 62.5\% of stocks tested, it proves to be very effective for certain stocks while the deep learning model is far more risk averse than any of the other models explored.


\end{document}
