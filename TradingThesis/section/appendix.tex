\documentclass[../thesis.tex]{subfiles}
\begin{document}
\chapter{Appendix}
\label{ch:append}

\subsection{Implementation}

Included in this section is all of the pseudocode for the implementations described above.

\subsubsection{SMA}

\begin{verbatim}
def execute(stock, start_date, end_date):
    stock = stockDataRetriever(stock, start_date, end_date).getStock()

    # Initialize the short and long windows and buy sell in df
    short_window = 40
    long_window = 100
    
    # set Pandas df with stock data
    df = pd.DataFrame(index=stock.index)

    # Create short and long simple moving average 
    df['short_moving_average'] = df.movingWindow(short_window)
    df['long_moving_average'] = df.movingWindow(long_window)
    
    # mark signal based on comparison of the two averages
    df['signal'] = df.compare(short_moving_average, long_moving_average,1,0)
    
    # when signal changes from 1 to 0 or 0 to 1 - is a buy or sell
    df['positions'] = df['signal'].diff()

\end{verbatim}

\subsubsection{EMA }

\begin{verbatim}
def execute(stock, start_date, end_date):
    stock = stockDataRetriever(stock, start_date, end_date).getStock()

    # Initialize the short and long windows and buy sell in df
    short_window = 12
    long_window = 26

    # set Pandas df with stock data
    df = pd.DataFrame(index=stock.index)

    # Create short and long simple moving average
    df[short_exponential_moving_average] = df.ema(short_window)
    df[long_exponential_moving_average] = df.ema(long_window)

    # mark signal based on comparison of the two averages
    df[signal] = df.compare(short_exponential_moving_average, long_exponential_moving_average,1,0)

    # when signal changes from 1 to 0 or 0 to 1 - is a buy or sell
    df[positions] = df[signal].diff()

\end{verbatim}

\subsubsection{Bollinger Bands }

\begin{verbatim}
def execute(stock, start_date, end_date):
    stock = stockDataRetriever(stock, start_date, end_date).getStock()

    # Initialize the window
    window = 12

    # set Pandas df with stock data
    df = pd.DataFrame(index=stock.index)

    # Create the bollinger bands
    df['middle_band] = df.sma(window)
    df['moving_deviation'] = df.std(window)
    df['upper_band'] = df['middle_band'] + df['moving_deviation'] * 2
    df['lower_band'] = df['middle_band'] - df['moving_deviation'] * 2

    # mark signal based on when price moves above or below lower or upper bands
    df[signal] = df.compare((df_closing_price, df_lower band), 
    		     (df_closing_price, df_upper_band),1,0)

    # when signal changes from 1 to 0 or 0 to 1 - is a buy or sell
    df[positions] = df[signal].diff()

\end{verbatim}

\subsubsection{RSI }

\begin{verbatim}
def execute(stock1, stock2, start_date, end_date):
    stock = stockDataRetriever(stock1, start_date, end_date).getStock()

    # set Pandas df with both stocks
    df = pd.DataFrame(index=stock.index)

    # create measures of up and down performance
    delta = stock['Close'].diff()
    roll_up = delta[delta > 0].rolling.mean()
    roll_down = delta[delta > 0].rolling.mean()
    
    # create RSI
    RS = roll_up/roll_down
    RSI = 100.0 - (100.0 / (1.0 + RS))

    # create the buy and sell commands
    df['buy'] = np.where(RSI < 30, 1, 0)
    df['sell'] = np.where(RSI > 70, -1, 0)
    df['signal'] = df['buy'] + df['sell']

    # when signal changes from 1 to 0 or 0 to 1 - is a buy or sell
    df[positions] = df[signal].diff()

\end{verbatim}

\subsubsection{Combination Momentum Strategies }

\begin{verbatim}
def execute(stock, start_date, end_date):
    stock = stockDataRetrieverIntraday(stock, start_date, end_date).getStock()
	
    # create df for both RSI and MACD and merge
    rsi_df = RSI(name, stock)
    macd_df = MACD(name, stock)
    df = pd.merge(rsi_df, macd_df)
    
    # buy signals when both rsi and macd indicate buy
    # sell signals when either one of rsi or macd indicate sell
    df['buy'] = (df['signal_rsi'] == 1) & (df['signal_macd'] == 1)
    df['sell'] = (df['signal_rsi'] == -1) | (df['signal_macd'] == -1)
    df['signal] = df['buy'] + df['sell']

\end{verbatim}

\subsubsection{Pairs Trading}

\begin{verbatim}
def execute(stock1, stock2, start_date, end_date):
    stock_1 = stockDataRetriever(stock1, start_date, end_date).getStock()
    stock_2 = stockDataRetriever(stock2, start_date, end_date).getStock()

    # set Pandas df with both stocks
    df = pd.DataFrame(index=stock_1.index, stock_2.index)

    # create zscore from price ratios
    ratios = df['Close_stock1']/df['Close_stock2']
    zscore = (ratios - ratios.mean())/ np.std(ratios)

    # Create the buy and sell commands
    df['buy'] = np.where(zscore < -1, 1, 0)
    df['sell'] = np.where(zscore > 1, -1, 0)
    df['signal'] = df['buy'] + df['sell']

    # when signal changes from 1 to 0 or 0 to 1 - is a buy or sell
    df[positions] = df[signal].diff()

\end{verbatim}

\end{document}
